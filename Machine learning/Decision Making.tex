\documentclass[11pt]{article}
% 쪽번호 생성
% \usepackage{plain}

\usepackage{fancyhdr}
% \usepackage{headings}

\usepackage{amsmath}
% use UTF8 encoding
\usepackage[utf8]{inputenc}
% use KoTeX package for Korean
\usepackage{kotex}

\usepackage{hyperref}

\usepackage{graphicx}

% \setlength{\headheight}{30pt}

% \pagestyle{headings}
\pagestyle{fancy}

\fancyhead[O]{output you want}

% \markright{John Smith\hfill On page styles\hfill}

\renewcommand{\headrulewidth}{4pt}
% \renewcommand{\footrulewidth}{2pt}

\fancyhf{}

% \fancyhead[L]{\includegraphics[width=1cm]{example-image-a}}
% \fancyhead[C]{}
% \fancyhead[R]{\rightmark}
% \fancyfoot[L]{ABC}
% \fancyfoot[C]{\textcopyright xyz}
% \fancyfoot[R]{\thepage}

% \hypersetup{
%     colorlinks=true,   
%     urlcolor=red,
% }

\title{Decision making}

\author{Minwoo Jung}

\begin{document}

\maketitle

\section{개요}
\indent \\본 문서는 스마트 가든 플랫폼의 기술문서이다. 본 문서는 스마트 가든 플랫폼에 대한 구성, 특성, 설치법 등 다양한 정보를 포함하고 있다. 

\subsection{시스템 특징}
\indent \\스마트 가든 플랫폼은 관개제어기, 게이트웨이, 서버로 구성되어 있다. 관개제어기는 토양 환경을 모니터링 할 수 있는 센서부, 토양 환경 변화를 감지하여 능동적으로 관개시스템을 제어할 수 있는 제어부, 외부충격 방지와 방수를 위한 기구부로 구성된다. 

\subsection{관개제어기}
\indent \\센서부는 영양분(NPK), 전기전도성(EC), 온습도, 산성도(pH)와 같은 토양센서를 포함하고 있다. 센서와 관개제어기는 RS-485 MODBUS 프로토콜을 기반 버스로 연결되어 있다. 관개제어기에 연결 가능한 센서 개수는 최대 4개이다. 센서에서 수집된 토양 정보는 게이트웨이로 전송된다.
관개제어기는 유비노스 운영체제가 탑재되어 있다. 유비노스 운영체제는  
관개제어기는 게이트웨이와 데이터 송수신을 위해서 블루투스(BLE) 프로토콜을 지원한다.

\subsubsection{하드웨어 구조}
\indent \\관개제어기의 하드웨어는 데이터 처리와 블루투스 통신을 위한 MCU, 솔레노이드 제어를 위한 Relay, 토양 센서 데이터 수집을 위한 RS-485 트랜시버로 구성되어 있다. 센서와 솔레노이드 전원은 DC-12V 입력을 인가한다. RS-485 트랜시버는 

\subsubsection{RS-485 통신환경}
\indent \\RS-485 통신은 송/수신을 동시에 할 수 없는 반이중(Half-duplex) 방식을 채택하고 있어서 트랜시버에 RX와 TX를 커트롤하는 핀이 있다. 또한 데이터 전송 거리는 최대 1.2km로 장거리 전송이 가능하다. 일반적으로 100kbps 전송 속도에서 약 1.2km 전송거리를 가지며, 10Mbps 전송 속도로 올리면 전송거리가 12m로 줄어든다. RS-485통신은 하나의 노드에 여러 대의 장비를 병렬로 연결할 수 있는 멀티드랍(Multi-drop)이 가능하다.

\begin{table}
    \centering
    \caption{Communication Basic Parameters}
    \label{t1}
    % \begin{tabular}{|c|c|c|c|}
    \resizebox{\textwidth}{!}
    {%
    \begin{tabular}{|c|c|}
    \noalign{\smallskip}\noalign{\smallskip}\hline
    % & column1 & column2 & column3 \\
    PARAMETERS & content \\
    % \hline
    % row1 & - & - & - \\
    % \hline
    % row2 & - & - & - \\
    % \hline
    % row3  & - & - & - \\
    % \hline
    \hline
    Protocol & Modbus RTU \\
    \hline
    Data bits & 8 bit \\
    \hline
    Parity  & No \\
    \hline
    Stop bit  & 1 bit \\
    \hline
    Error checking  & CRC (redundant loop code) \\
    \hline
    Baud rate  & 2400 bps/ 4800 bps/ 9600 bps can be set factory defaults to 9600 bps\\
    \hline
    \end{tabular}
    }
\end{table}

\begin{table}
    \centering
    \caption{Register Address}
    \label{t2}
    \resizebox{\textwidth}{!}
    {
    \begin{tabular}{|c|c|c|c|}
    % \resizebox{\textwidth}{!}
    % {
    % \begin{tabular}{|c|c|}
    \noalign{\smallskip}\noalign{\smallskip}\hline
    % & column1 & column2 & column3 \\
    Register Address & PIC Configuration Address & content & Operation \\
    % \hline
    % row1 & - & - & - \\
    % \hline
    % row2 & - & - & - \\
    % \hline
    % row3  & - & - & - \\
    % \hline
    \hline
    0002H & 40003 & Soil humidity & Read-Only \\
    \hline
    0003H & 40004 & Soil temperature & Read-Only \\
    \hline
    0100H & 40101 & Device Address & Read-Write \\
    \hline
    0101H & 40102 & Baud Rate & Read-Write \\
    \hline    
    \end{tabular}
    }
\end{table}

% \begin{table}
%     \centering
%     \caption{Message Frame}
%     \label{t3}
%     % \resizebox{\textwidth}{!}
%     % {
%     \begin{tabular}{|c|c|c|c|c|c|}
%     % \resizebox{\textwidth}{!}
%     % {
%     % \begin{tabular}{|c|c|}
%     \noalign{\smallskip}\noalign{\smallskip}\hline
%     % & column1 & column2 & column3 \\
%     Inquiry Frame & & & & & &\\
%     % \hline
%     % row1 & - & - & - \\
%     % \hline
%     % row2 & - & - & - \\
%     % \hline
%     % row3  & - & - & - \\
%     % \hline
%     \hline
%     Address Code & Function Code & Start Address & Data Length & CRC_L & CRC_H \\
%     \hline
%     Answer Frame & & & & & & \\
%     \hline
%     Address Code & Function Code & Number of Valid Bytes & Value & CRC_L & CRC_H \\    
%     \hline
%     \end{tabular}
%     % }
% \end{table}

\subsubsection{메시지구조}
\indent \\MODBUS는 산업용 표준 직렬 통신 프로토콜이다. MODBUS는 마스트와 슬레이브 프레임워크로 구성되어 있다. 슬레이브는 고유 주소를 가진다. 토양 센서에서 제공하는 MODBUS 기본 파라미터는 표\ref{t1}과 같다. 표\ref{t2}는 센서의 레지스터 주소값을 보여주고 있다. 레지스터 주소값을 기반으로 센싱 데이터를 요청할 수 있고, 센서 고유 아이디와 전송속도를 변경할 수 있다. %표\ref{t3}는 request message와 respond message 구조를 보여주고 있다.%
 

\subsection{게이트웨이}
\indent \\게이트웨이는 다수의 관개제어기로부터 토양 데이터를 수집하고, MQTT 프로토콜을 기반으로 서버로 데이터를 전송한다. 게이트웨이의 멀티태스킹을 위해서 유비노스 운영체제를 탑재하였다. 

\subsection{서버}
\indent \\서버는 MQTT(Message Queuing Telemetry Transport) 브로커와 데이터베이스로 구성되어 있다. MQTT는 publish/subscribe/broker를 사용하는 프로토콜로서 subscriber는 토픽을 구독하기 위한 목적,publisher는 토픽을 발행하기 위한 목적으로 broker와 연결된다. 

\end{document}