\documentclass[11pt]{article}
% 쪽번호 생성
% \usepackage{plain}

\usepackage{fancyhdr}
% \usepackage{headings}

\usepackage{amsmath}
% use UTF8 encoding
\usepackage[utf8]{inputenc}
% use KoTeX package for Korean
\usepackage{kotex}

\usepackage{hyperref}

\usepackage{graphicx}

% \setlength{\headheight}{30pt}

% \pagestyle{headings}
\pagestyle{fancy}

\fancyhead[O]{output you want}

% \markright{John Smith\hfill On page styles\hfill}

\renewcommand{\headrulewidth}{4pt}
% \renewcommand{\footrulewidth}{2pt}

\fancyhf{}

% \fancyhead[L]{\includegraphics[width=1cm]{example-image-a}}
% \fancyhead[C]{}
% \fancyhead[R]{\rightmark}
% \fancyfoot[L]{ABC}
% \fancyfoot[C]{\textcopyright xyz}
% \fancyfoot[R]{\thepage}

% \hypersetup{
%     colorlinks=true,   
%     urlcolor=red,
% }

\title{Hall sensor}

\author{Minwoo Jung}

\begin{document}

\maketitle

\section{서론}
\indent \\하이트레벨센서는 차량의 편의성과 안전성을 제공하기 위해서 섀시 제어 및 헤드라이트 빔 조정 기능을 구현하는데 반드시 필요하다.
\indent \\홀센서는 전류가 흐르는 도체에 자기장이 가해질 때 발생하는 홀효과를 이용하여 자기장의 세기와 극성을 측정하는 원리로 동작한다.
\indent \\차량 안전 시스템에 사용될 수 있으며, 차량의 안정성과 조작성을 향상시키는 데 중요한 역할을 합니다. 높낮이 센서는 일반적으로 차량의 하부에 부착되며, 차량의 높이를 측정하는데 사용되는 초음파, 레이저, 또는 광선 센서 등을 포함할 수 있습니다. 이러한 센서는 보통 차량의 전방, 후방, 좌우측 아래에 위치하며, 주행 중에도 차량의 높이를 측정할 수 있습니다. 높낮이 센서는 차량의 높이를 측정하여, 불규칙한 지형이나 높은 장애물 등과의 충돌을 방지하고, 주차장에서의 주차를 쉽게 할 수 있도록 도와줍니다. 또한, 자율주행차량의 경우, 높낮이 센서를 사용하여 도로 상황에 따라 차량의 높이를 조정하고, 안정적인 주행을 유지할 수 있습니다.

차량의 높이 센서는 여러 가지 이유로 중요합니다. 차량 높이 센서의 주요 요구 사항 중 일부는 다음과 같습니다.

안전: 높이 센서는 차량과 탑승자의 안전을 보장하는 데 도움이 됩니다. 지면 위의 차량 높이를 측정함으로써 센서는 범프, 장애물 또는 가파른 경사와 같은 잠재적인 위험을 감지할 수 있습니다. 그런 다음 이 정보를 사용하여 차량의 서스펜션 또는 기타 시스템을 조정하여 안전하고 원활한 주행을 보장할 수 있습니다.

효율성: 높이 센서는 차량의 효율성도 향상시킬 수 있습니다. 예를 들어 차량의 지상고를 조정하여 공기역학을 개선하고 항력을 줄여 연비를 개선하는 데 사용할 수 있습니다.

주차: 주차 시 운전자를 돕기 위해 높이 센서를 사용할 수 있습니다. 차량과 지면 사이의 거리를 감지할 수 있어 운전자가 주차하는 동안 연석이나 기타 장애물에 부딪히는 것을 방지할 수 있습니다.

오프로드: 오프로드 차량의 경우 높이 센서는 고르지 않은 지형에서 차량의 각도를 감지하고 이에 따라 서스펜션 및 기타 시스템을 조정하여 안정성과 견인력을 유지할 수 있으므로 높이 센서가 특히 중요합니다.

전반적으로 높이 센서는 안전, 효율성 및 전반적인 성능을 개선하는 데 도움이 되므로 최신 차량의 중요한 구성 요소입니다.


\indent \\높이 감지 기술을 서스펜션 컨트롤 암에 통합하면 차량의 움직임과 주변 환경에 대한 귀중한 데이터를 수집할 수 있습니다. 이 데이터는 예측 유지 보수, 도로 상태 모니터링 및 전자 안전 시스템 최적화와 같은 다양한 목적으로 분석되고 사용될 수 있습니다. 스마트 섀시 구성 요소는 또한 새로운 비즈니스 모델을 지원하고 차량 비즈니스의 투명성을 높일 수 있습니다. 향후 추가 센서 기능을 통합함으로써 렌트 차량의 기술적 조건을 문서화할 수 있으며, 이는 다양한 사용자가 차량을 운영하는 새로운 모빌리티 개념에 특히 중요합니다. 수집된 데이터는 도로 표면의 균열이나 움푹 들어간 곳과 같은 취약 지점을 보다 빠르고 효율적으로 식별하여 도로 인프라 유지 관리에도 도움이 될 수 있습니다.

\section{관련연구}

\section{선형알고리즘}


\end{document}